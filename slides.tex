\documentclass[xcolor=dvipsnames]{beamer}
\usepackage{listings}
\usepackage[utf8]{inputenc}
\usetheme{Warsaw}

% www.parashift.com/c++-faq-lite/latex-macros.html
\newcommand{\CC}{C\nolinebreak\hspace{-.05em}\raisebox{.4ex}{\scriptsize\bf +}\nolinebreak\hspace{-.05em}\raisebox{.4ex}{\scriptsize\bf +}}

\title[C\raisebox{.2ex}{++}11 tour\hspace{25mm} \insertframenumber/\inserttotalframenumber]{\CC{}11 tour\\What will change for library designers?}
\author[Maxime van Noppen]{Maxime van Noppen\\maxime.van.noppen@gmail.com\\\url{https://github.com/mvannop/cpp11_tour}}
\date{24 July 2012}
\begin{document}

%\definecolor{lightgrey}{gray}{0.8}

\lstdefinelanguage{cpp}[]{C++}%
  {
    deletekeywords={bool,wchar_t,char,double,float,int,long,short,signed,unsigned,void},
    classoffset=0,
    keywordstyle=\color{orange},
    classoffset=1,
    morekeywords={bool,wchar_t,char,double,float,int,long,short,signed,unsigned,void},
    keywordstyle=\color{blue},
    classoffset=0,
    basicstyle=\footnotesize\ttfamily,
    commentstyle=\color{red},
    showspaces=false,
    showstringspaces=false
  }%

\AtBeginSection[]
{
  \begin{frame}{Outline}
    \tableofcontents[currentsection,currentsubsection]
  \end{frame}
} 


\begin{frame}
\titlepage
\end{frame}

\begin{frame}{Table of contents}
\tableofcontents
\end{frame}

\section{Big picture}
\subsection{What could change?}
\begin{frame}
\frametitle{The big picture: what could change?}

Adopting \CC{}11 should hopefully help us to:

\begin{itemize}
  \item Improve maintainability: less overhead, less bloat
  \pause
  \item Improve performances: new ways to manipulate objects
  \pause
  \item Create new designs enabling new features: new OO and generic language features
\end{itemize}

\pause
At the end of the day:
\begin{itemize}
  \item Development costs should decrease for the library developer \textbf{and} the library user
  \pause
  \item Product value should increase
\end{itemize}
\end{frame}

\subsection{Work in progress}
\begin{frame}[fragile]
\frametitle{The big picture: Work in progress}

\begin{itemize}
  \item Fully-compliant compilers not expected before 2013\ldots
  \pause
  \item \ldots but recent compilers (g++ 4.7, llvm 3.1, MSVC 11) are moving fast\ldots
  \pause
  \item \ldots and this is great compared to \CC{}98/\CC{}03 compilers lag.
  \pause
  \item {\footnotesize \url{https://wiki.apache.org/stdcxx/C%2B%2B0xCompilerSupport}}
\end{itemize}

\pause

\begin{itemize}
  \item Using \CC{}11 feels like exploring a new language
  \pause
  \item No "best practices" set in stone yet
  \pause
  \item \CC{}1x (\CC{}17?) is already in the tubes
\end{itemize}
\end{frame}

\section{Rvalue references}
\begin{frame}[fragile]
  \frametitle{Problem}

  \lstinputlisting[language=cpp]{src/problem.cc}

  \begin{itemize}
    \item Copying a temporary might be very expensive
    \item No easy solution
    \item We have to hope that "return value optimisation" (RVO) kicks in
    \item What happens with complex formulas mixing a lot of operators?
  \end{itemize}
\end{frame}
\begin{frame}[fragile]
  \begin{itemize}
    \item Avoid deep-copying temporaries
    \pause
    \item More reliable than "return value optimisation" (RVO)
    \pause
    \item Eases writing of overloaded operators on complex types
  \end{itemize}

  \lstinputlisting[language=cpp]{src/rvalue_ref_01.cc}
  \pause
  \begin{itemize}
    \item Simply recompiling code with a move-enabled STL may increase performances
  \end{itemize}
\end{frame}

\begin{frame}[fragile]
  \begin{itemize}
    \item Create a move constructor
  \end{itemize}

  \lstinputlisting[language=cpp]{src/rvalue_ref_02.cc}
\end{frame}

\section{Variadic templates}
\begin{frame}[fragile]
  \frametitle{Problem}
  What is a \textbf{generic} function?
  \pause
  \begin{itemize}
    \item In \CC{}03: generic on the \textbf{type} of its arguments\\
      \begin{lstlisting}[language=cpp]
        template <typename T> void foo(T t) { }
      \end{lstlisting}
    \pause
    \item But not generic on the \textbf{number} of its arguments
    \pause
    \item What about the good old printf-style functions?
  \end{itemize}
  \pause
  Challenge: write a 'min' function which takes an arbitrary number of arguments
  of any type and returns the smallest in \CC{}03.\\
  \pause
  \vspace{0.5cm}
  To support up to \verb#N# arguments there needs to be \verb#N# overloads\ldots
\end{frame}

\begin{frame}[fragile]
  \frametitle{Introducing variadic templates}
  \begin{itemize}
    \item Combination of the old '\ldots' (think of printf) and classic templates
    \pause
    \item What about our challenge?
    \pause
  \end{itemize}

  \lstinputlisting[language=cpp]{src/min.cc}

  \begin{itemize}
    \item Very efficient thanks to inlining
  \end{itemize}
\end{frame}

\begin{frame}[fragile]
    
  \frametitle{A typesafe printf!}

  \lstinputlisting[language=cpp]{src/variadic_01.cc}
\end{frame}

\section{Rvalue references + variadic templates}
\begin{frame}[fragile]
  \frametitle{Problem}
  Writing a generic "make\_shared" function which \textbf{constructs} the object.\\
  Either:
  \begin{itemize}
    \item Create \textbf{a lot} of overloads\ldots
      \pause
    \item Add constraints on the type, typically being default constructible
      \pause
    \item \ldots don't construct the object but take a pointer to an already constructed object\ldots
  \end{itemize}

  \lstinputlisting[language=cpp]{src/boost_shared.cc}

\end{frame}
\begin{frame}[fragile]
  \frametitle{Perfect forwarding to the rescue!}
  \begin{itemize}
    \item Rvalue references + variadic templates = perfect forwarding
  \end{itemize}

  \lstinputlisting[language=cpp,basicstyle=\scriptsize]{src/make_shared.cc}
  \pause
  \begin{itemize}
    \item Used in the STL to create emplace\_* methods
  \end{itemize}

  \lstinputlisting[language=cpp]{src/emplace.cc}
\end{frame}

\section{Conclusion}
\begin{frame}
  \frametitle{Conclusion}

  \emph{"The pieces just fit together better than they used to and I find a
  higher-level style of programming more natural than before and as efficient as ever"}
  --- Bjarne Stroustrup {\footnotesize \url{http://www2.research.att.com/~bs/C++0xFAQ.html}}\\
  \vspace{0.2cm}
  \only<2>{As library designers:}
  \only<3>{As library users:}
  \only<2-3>{
  \begin{itemize}
    \item More powerful and safe interfaces
    \item Better performances with simpler designs
    \item And this with a better maintainability 
  \end{itemize}}
  \only<4>{
  There is much more!
  \begin{itemize}
    \item We only scratched the surface of \CC{}11
    \item Code bloat should decrease while features should increase 
    \item We will see new paradigms emerge from the new core features
  \end{itemize}}
\end{frame}

\begin{frame}
  \frametitle{Bibliography}
  \begin{itemize}
    \item \CC{} Now! (former BoostCon) \url{http://cppnow.org}
      \begin{itemize}
        \item \textbf{Leor Zolman}: A Whirlwind Overview of \CC{}11
        \item \textbf{Alisdair Meredith}: Lessons Learned Developing the \CC{}11 Standard Library
        \item \textbf{Howard Hinnant}: What’s new with \CC{}11 containers?
        \item \textbf{Scott Schurr}: \CC{}11: New Tools for Class and Library Authors
        \item \url{http://cppnow.org/schedule-table/}
      \end{itemize}
    \item \textbf{Bjarne Stroustrup}: {\footnotesize \url{http://www2.research.att.com/~bs/}}
    \item \textbf{\CC{}Next}: \url{http://cpp-next.com/}
  \end{itemize}
\end{frame}

\begin{frame}
  \begin{center}
    Questions?
  \end{center}
\end{frame}

\begin{frame}[fragile, noframenumbering]
  \frametitle{Mixins}
  \lstinputlisting[language=cpp]{src/mixins.cc}
\end{frame}

\begin{frame}[fragile]
  \frametitle{Type inference}
  
  \lstinputlisting[language=cpp]{src/type_inf.cc}
\end{frame}

\end{document}
